\documentclass[t]{beamer}
\usepackage[utf8]{inputenc}  % to be able to type unicode text directly
\usepackage{inconsolata}     % for a nicer (e.g. non-courier) tt family font
\usepackage{array}           % to fine-tune tabular spacing
\usepackage{bbm}             % for blackboard 1
\usepackage{graphicx}        % to include images
\usepackage{soul}            % for colored strikethrough

\colorlet{darkgreen}{black!50!green}  % used for page numbers
\definecolor{term}{rgb}{.9,.9,.9}     % used for code insets

\setlength{\parindent}{0em}  % no paragraph indentation
\setlength{\parskip}{1em}    % paragraph spacing


% coco's macros
\def\R{\mathbf{R}}
\def\F{\mathcal{F}}
\def\x{\mathbf{x}}
\def\y{\mathbf{y}}
\def\u{\mathbf{u}}
\def\Z{\textbf{Z}}
\def\d{\mathrm{d}}
\DeclareMathOperator*{\argmin}{arg\,min}
\DeclareMathOperator*{\argmax}{arg\,max}
\newcommand{\reference}[1] {{\scriptsize \color{gray}  #1 }}
\newcommand{\referencep}[1] {{\tiny \color{gray}  #1 }}
\newcommand{\unit}[1] {{\tiny \color{gray}  #1 }}

% disable spacing around verbatim
\usepackage{etoolbox}
\makeatletter\preto{\@verbatim}{\topsep=0pt \partopsep=0pt }\makeatother

% disable headings, set slide numbers in green
\mode<all>\setbeamertemplate{navigation symbols}{}
\defbeamertemplate*{footline}{pagecount}{\leavevmode\hfill\color{darkgreen}
   \insertframenumber{} / \inserttotalframenumber\hspace*{2ex}\vskip0pt}

%% select red color for strikethrough
\makeatletter
\newcommand\SoulColor{%
  \let\set@color\beamerorig@set@color
  \let\reset@color\beamerorig@reset@color}
\makeatother
\newcommand<>{\St}[1]{\only#2{\SoulColor\st{#1}}}
\setstcolor{red}

% make everything monospace
\renewcommand*\familydefault{\ttdefault}

\begin{document}

\addtocounter{framenumber}{-1}
\begin{frame}[plain,fragile]
\LARGE
\begin{verbatim}





  visualization of topographic maps




mnhrdt
gtti 25/5/2022
\end{verbatim}
\end{frame}

\begin{frame}
VISUALIZATION OF TOPOGRAPHIC MAPS\\
=================================

A {\bf topographic map} is a function~$\color{blue}(x,y)\mapsto h(x,y)$
representing the height of a surface.

\vfill

\vfill

{\bf\color{darkgreen} The best way} to visualize a topographic map is
to rotate the 3D surface interactively.
\pause

\fbox{
\fbox{
\fbox{
My goal here is to present {\bf\color{red} the second best} way.
}
}
}

\pause
* \St{How to use GDAL/QGIS to create hill-shaded maps?}\\
* What do GDAL/QGIS do to create hill-shaded maps?
\end{frame}

%\begin{frame}[fragile]
%VISUALIZATION OF TOPOGRAPHIC MAPS\\
%=================================
%
%
%\tiny
% \begin{verbatim}
%0. History, context
%0.1. What is a TM, DEM, DSM, DTM
%0.2. History of map colorization
%0.3. Fuji 3d photos, identify landmarks
%0.4. Rotate 3d model interactively
%
%1. Symbolic-based methods
%1.1. Palette (gray, color)
%1.2. Level lines
%1.3. Level lines + palette (zelda!)
%1.4. Slope = closeness of level lines, norm of gradient
%1.5. Slope angle = direction of level lines, gradient vector field, grad.-form
%1.6. Curvature sign = convexity, concavity of level lines
%
%2. Light-based methods
%2.1. Shadows
%2.2. Lambertian shading (gouraud, + linear approximation)
%2.3. Clear sky shading (ao sampling, + linear approximation)
%2.4. Specular shading (phong)
%2.5. SAR direct model
%
%3. Uses and fancy things
%3.1. To observe defects, understand geometry, etc
%3.2. Combined methods: tinted hillshade with level-lines
%3.3. As a texture for a 3D rendering
%3.4. As a direct model for several famous inverse problems
%3.5. Take home message: do not show "north up" in optical images
%3.6. Colophon: this document is a notebook
%\end{verbatim}
%\end{frame}

\begin{frame}
HISTORY\\
=======

Leonardo da Vinci's maps:

\begin{tabular}{ll}
	\includegraphics[width=0.5\linewidth]{fs/centralitaly.png}&
	\includegraphics[width=0.5\linewidth]{fs/tuscany2.jpg}\\
	\small central Italy, 1502 &
	\small Tuscany, 1503 \\
	\small (color-coded height) &
	\small (hill-shading)
\end{tabular}
\end{frame}

\begin{frame}
HISTORY\\
=======

Hans-Conrad Gyger's military map of Switzerland, 1668

\begin{tabular}{ll}
	\includegraphics[width=0.45\linewidth]{fs/pgyger.jpg}&
	\includegraphics[width=0.5\linewidth]{fs/zurich.png}\\
	\small one tile &
	\small detail around Zurich \\
\end{tabular}

\vfill\pause\scriptsize
\color{blue}
Published in 1880, this is still today the ``gold standard'' of mapping.
\end{frame}

%\begin{frame}
%What is a TM, DEM, DSM, DTM\\
%===========================
%\end{frame}

\begin{frame}
MODERN CARTOGRAPHY\\
==================

\vfill

\begin{tabular}{ll}
	\includegraphics[width=0.475\linewidth]{fs/europe.jpg}&
	\includegraphics[width=0.5\linewidth]{fs/dalmacija.jpg}\\
\end{tabular}

Modern tools like QGIS, GDAL, ArcGis, etc. offer map renderers based on
ray-tracing techniques from computer graphics.  Everything is possible.
\end{frame}

\begin{frame}
MODERN CARTOGRAPHY\\
==================

\includegraphics[width=\linewidth]{fs/satire.png}
\end{frame}

\begin{frame}
OUR TARGETS: FUJI AND OMURO\\
===========================


\includegraphics[width=0.4\linewidth]{fs/fuji_srtm4.png}SRTM4
\vfill

\begin{tabular}{ll}
	\includegraphics[width=0.45\linewidth]{fs/fujiview.jpg}&
	\includegraphics[width=0.45\linewidth]{fs/omuroview.jpg}\\
	Fuji &
	Omuro
\end{tabular}

\end{frame}



\begin{frame}
OVERVIEW\\
========

\footnotesize
{\color{gray}0. History\\
}

\bigskip

1. Symbolic methods\\
1.1. Palette\\
1.2. Level-lines\\
1.3. Slope\\
1.4. Gradient\\
1.5. Laplacian, curvature\\

\bigskip
{
\color{gray}
2. Light-based methods\\
2.1. Shadows\\
2.2. Point-source shading (exact, linear approx.)\\
2.3. Cloudy-sky shading (exact, linear approx.)\\
2.4. Specular shading (Phong)\\
2.5. Reflective shading (Oren-Nayar)\\
2.6. SAR direct model\\
}

\bigskip

{\color{gray}
3. Fancier stuff
}
%0. History, context
%0.1. What is a TM, DEM, DSM, DTM
%0.2. History of map colorization
%0.3. Fuji 3d photos, identify landmarks
%0.4. Rotate 3d model interactively
%
%1. Symbolic-based methods
%1.1. Palette (gray, color)
%1.2. Level lines
%1.3. Level lines + palette (zelda!)
%1.4. Slope = closeness of level lines, norm of gradient
%1.5. Slope angle = direction of level lines, gradient vector field, grad.-form
%1.6. Curvature sign = convexity, concavity of level lines
%
%2. Light-based methods
%2.1. Shadows
%2.2. Lambertian shading (gouraud, + linear approximation)
%2.3. Clear sky shading (ao sampling, + linear approximation)
%2.4. Specular shading (phong)
%2.5. SAR direct model
%
%3. Uses and fancy things
%3.1. To observe defects, understand geometry, etc
%3.2. Combined methods: tinted hillshade with level-lines
%3.3. As a texture for a 3D rendering
%3.4. As a direct model for several famous inverse problems
%3.5. Take home message: do not show "north up" in optical images
%3.6. Colophon: this document is a notebook

\end{frame}

%\begin{frame}
%Fuji 3d photos, identify landmarks\\
%================
%
%% https://earth.google.com/web/@35.43358604,138.66641909,1286.03715416a,6726.59176094d,35y,115.23907572h,78.15012473t,0r
%\end{frame}
%
%\begin{frame}
%Rotate 3d model interactively\\
%================
%\end{frame}

%\begin{frame}
%Symbolic-based methods\\
%================
%\end{frame}

\begin{frame}[fragile]
PALETTES: GRAY, COLOR, PERIODIC, TOPOGRAPHIC\\
============================================

%SCRIPT palette 0 4000 gray f/fuji.npy f/fuji_gray.png -l f/fuji_gray_l.png &
%SCRIPT palette 0 4000 dem f/fuji.npy f/fuji_dem.png -l f/fuji_dem_l.png &
%SCRIPT palette 0 4000 fs/Default.gpl f/fuji.npy f/fuji_def.png -l f/fuji_def_l.png &
%SCRIPT palette 0 4000 botw f/fuji.npy f/fuji_botw.png -l f/fuji_botw_l.png &

%SCRIPT plambda f/fuji.npy "x 270 + 360 / 1 fmod 360 *  1 1 rgb hsv2rgb 255 *"|blur C 1|qauto -i - f/fuji_franges.png &

%SCRIPT plambda f/fuji.npy 'x,n not x 21 < or 255 *'|morsi square opening - f/fuji_water.png
%SCRIPT plambda f/fuji_water.png f/fuji_dem.png "x not y 20 100 255 rgb if" -o f/fuji_dem_water.png &
\only<1>{\includegraphics[width=0.9\textwidth]{f/fuji_gray.png}%
\includegraphics[width=0.1\textwidth]{f/fuji_gray_l.png}}%
\only<2>{\includegraphics[width=0.9\textwidth]{f/fuji_def.png}%
\includegraphics[width=0.1\textwidth]{f/fuji_def_l.png}}%
\only<3>{\includegraphics[width=0.9\textwidth]{f/fuji_franges.png}}
\only<4>{\includegraphics[width=0.9\textwidth]{f/fuji_dem.png}%
\includegraphics[width=0.1\textwidth]{f/fuji_dem_l.png}}%
\only<5>{\includegraphics[width=0.9\textwidth]{f/fuji_dem_water.png}%
\includegraphics[width=0.1\textwidth]{f/fuji_dem_l.png}}%
%\only<6>{\includegraphics[width=0.9\textwidth]{f/fuji_botw.png}%
%\includegraphics[width=0.1\textwidth]{f/fuji_botw_l.png}}%

\end{frame}

\begin{frame}
LEVEL LINES\\
===========

%SCRIPT blur -s l 1 f/fuji.npy f/fuji_b.npy

%SCRIPT echo 1000 500 400 300 200 100 90 80 70 60 50 40 30 20 10|tr ' ' '\n'|while read s; do plambda f/fuji_b.npy "$s >1 <1 / floor <1 *"|morsi cross egradient|plambda '-1 * dup dup rgb'|qauto -p 0|fontu puts 10x20 10 10 "S=$s" -c 800 -b 888 - f/fuji_ll_$s.png ; done &

\only<1>{\includegraphics[width=0.9\textwidth]{f/fuji_ll_1000.png}}%
\only<2>{\includegraphics[width=0.9\textwidth]{f/fuji_ll_500.png}}%
\only<3>{\includegraphics[width=0.9\textwidth]{f/fuji_ll_400.png}}%
\only<4>{\includegraphics[width=0.9\textwidth]{f/fuji_ll_300.png}}%
\only<5>{\includegraphics[width=0.9\textwidth]{f/fuji_ll_200.png}}%
\only<6>{\includegraphics[width=0.9\textwidth]{f/fuji_ll_100.png}}%
%\only<7>{\includegraphics[width=0.9\textwidth]{f/fuji_ll_90.png}}%
%\only<8>{\includegraphics[width=0.9\textwidth]{f/fuji_ll_80.png}}%
%\only<9>{\includegraphics[width=0.9\textwidth]{f/fuji_ll_70.png}}%
%\only<10>{\includegraphics[width=0.9\textwidth]{f/fuji_ll_60.png}}%
%\only<11>{\includegraphics[width=0.9\textwidth]{f/fuji_ll_50.png}}%
%\only<12>{\includegraphics[width=0.9\textwidth]{f/fuji_ll_40.png}}%
%\only<13>{\includegraphics[width=0.9\textwidth]{f/fuji_ll_30.png}}%
%\only<14>{\includegraphics[width=0.9\textwidth]{f/fuji_ll_20.png}}%
%\only<15>{\includegraphics[width=0.9\textwidth]{f/fuji_ll_10.png}}%
\end{frame}

\begin{frame}
LEVEL LINES + PALETTE\\
=====================

%SCRIPT plambda f/fuji_b.npy "200 / round 200 *" -o f/fuji_q.npy
%SCRIPT palette 0 4000 botw f/fuji_q.npy f/fuji_q.png
%SCRIPT morsi cross igradient f/fuji_q.npy | plambda f/fuji_q.png - 'x y not *'|plambda f/fuji_water.png - 'x not y 54 64 70 rgb if'|blur -s l .6|qauto -p 3 -i - f/fuji_zelda.png &
\only<1>{\includegraphics[width=0.9\textwidth]{f/fuji_zelda.png}%
\includegraphics[width=0.1\textwidth]{f/fuji_botw_l.png}}%
\only<2>{\includegraphics[width=0.9\textwidth]{f/fuji_zelda.png}%
\includegraphics[width=0.1\textwidth]{fs/wlink.png}}%
\end{frame}

\begin{frame}
DENSITY OF LEVEL LINES = SLOPE\\
==============================

%SCRIPT plambda f/fuji_b.npy "x,n -1 *" |qauto -p 0 - f/fuji_slope.png &

\only<1>{\includegraphics[width=0.9\textwidth]{f/fuji_ll_1000.png}$\color{blue}u^{-1}(1000\Z)$}%
\only<2>{\includegraphics[width=0.9\textwidth]{f/fuji_ll_500.png}$\color{blue}u^{-1}(500\Z)$}%
\only<3>{\includegraphics[width=0.9\textwidth]{f/fuji_ll_400.png}$\color{blue}u^{-1}(400\Z)$}%
\only<4>{\includegraphics[width=0.9\textwidth]{f/fuji_ll_300.png}$\color{blue}u^{-1}(300\Z)$}%
\only<5>{\includegraphics[width=0.9\textwidth]{f/fuji_ll_200.png}$\color{blue}u^{-1}(200\Z)$}%
\only<6>{\includegraphics[width=0.9\textwidth]{f/fuji_ll_100.png}$\color{blue}u^{-1}(100\Z)$}%
\only<7>{\includegraphics[width=0.9\textwidth]{f/fuji_ll_90.png}$\color{blue}u^{-1}(90\Z)$}%
\only<8>{\includegraphics[width=0.9\textwidth]{f/fuji_ll_80.png}$\color{blue}u^{-1}(80\Z)$}%
\only<9>{\includegraphics[width=0.9\textwidth]{f/fuji_ll_70.png}$\color{blue}u^{-1}(70\Z)$}%
\only<10>{\includegraphics[width=0.9\textwidth]{f/fuji_ll_60.png}$\color{blue}u^{-1}(60\Z)$}%
\only<11>{\includegraphics[width=0.9\textwidth]{f/fuji_ll_50.png}$\color{blue}u^{-1}(50\Z)$}%
\only<12>{\includegraphics[width=0.9\textwidth]{f/fuji_ll_40.png}$\color{blue}u^{-1}(40\Z)$}%
\only<13>{\includegraphics[width=0.9\textwidth]{f/fuji_ll_30.png}$\color{blue}u^{-1}(30\Z)$}%
\only<14>{\includegraphics[width=0.9\textwidth]{f/fuji_ll_20.png}$\color{blue}u^{-1}(20\Z)$}%
\only<15>{\includegraphics[width=0.9\textwidth]{f/fuji_ll_10.png}$\color{blue}u^{-1}(10\Z)$}%
%\only<16>{\includegraphics[width=0.9\textwidth]{f/fuji_ll_1.png}$\color{blue}u^{-1}(\Z)$}%
\only<16>{\includegraphics[width=0.9\textwidth]{f/fuji_slope.png}$\color{blue}\ \left\|\nabla u\right\|$}%

\end{frame}

\begin{frame}
FUNCTIONS AND DERIVATIVES\\
=========================

\vfill\tiny
\begin{tabular}{l|l|l}
	expression & meaning(surface) & meaning(level lines)\\
	&&\\
	\hline
	&&\\
$\color{blue}u(x,y)$ & height at point $(x,y)$ & \\
	&&\\
$\color{blue}u(x,y)=c$ &  & level line at height~$c$  \\
	&&\\
$\color{blue}\|\nabla u\|=\sqrt{u_x^2+u_y^2}$ & slope & density\\
	&&\\
$\color{blue}{\nabla u}/{\|\nabla u\|}$ & direction of maximum ascent & normal \\
	&&\\
$\color{blue}\Delta u = u_{xx} + u_{yy}$ & laplacian & ``convexity'' \\
	&&\\
$\color{blue}\mathrm{div}({\nabla u}/{\|\nabla u\|})$ & & curvature of level lines \\
\end{tabular}
\vfill
\end{frame}

\begin{frame}
GRADIENT \hfill{\color{blue}\fbox{$\nabla u = (u_x, u_y)$}}\\
========

%SCRIPT plambda f/fuji_b.npy 'x,g'|flowarrows 2 17 - f/fuji_arrows.png &
%SCRIPT plambda f/fuji_b.npy 'x,g'|viewflow 0 - f/fuji_viewflow.png &
%SCRIPT plambda f/fuji.npy 'x,gf split dup rgb'|qauto -p 1 - f/fuji_grad.png &
%SCRIPT flowarrows .1 23 fs/colorwheel.tiff f/colorwheel_arrows.png &
%SCRIPT viewflow 0 fs/colorwheel.tiff f/colorwheel_viewflow.png &
%SCRIPT plambda fs/colorwheel.tiff "split dup rgb"|qauto -p 0 - f/colorwheel_rgb.png &

\only<1>{\includegraphics[width=0.9\textwidth]{f/fuji_arrows.png}%
\includegraphics[width=0.1\textwidth]{f/colorwheel_arrows.png}\\arrows}%
\only<2>{\includegraphics[width=0.9\textwidth]{f/fuji_viewflow.png}%
\includegraphics[width=0.1\textwidth]{f/colorwheel_viewflow.png}\\colorwheel}%
\only<3>{\includegraphics[width=0.9\textwidth]{f/fuji_grad.png}%
\includegraphics[width=0.1\textwidth]{f/colorwheel_rgb.png}\\RGB=$(u_x,u_y,u_y)$}%

\end{frame}

\begin{frame}
LAPLACIAN \hfill{\color{blue}\fbox{$\Delta u = u_{xx}+ u_{yy}$}}\\
=========

%SCRIPT plambda f/fuji_b.npy 'x,l -1 *'|palette 1% 0 gray - f/fuji_glap.png -l f/fuji_glap_pal.png &
%SCRIPT plambda f/fuji_b.npy 'x,l -1 *'|blur c .5|PLEGEND_REVERSE=1 palette -10 10 nice - f/fuji_lap.png -l f/fuji_lap_pal.png &

\only<1>{
\includegraphics[width=0.9\textwidth]{f/fuji_glap.png}%
\includegraphics[width=0.1\textwidth]{f/fuji_glap_pal.png}
}%
\only<2>{
\includegraphics[width=0.9\textwidth]{f/fuji_lap.png}%
\includegraphics[width=0.1\textwidth]{f/fuji_lap_pal.png}\\
{\color{blue}valleys:$\Delta u\!>\!0\quad$}
{\color{red}ridges:$\Delta u\!<\!0\quad$}
{\color{gray}planar:$\Delta u\!\approx\!0$}}

\end{frame}


\begin{frame}
CURVATURE \hfill{\color{blue}\fbox{$\mathrm{curv}(u)=\mathrm{div}(\nabla u/\|\nabla u\|)$}}\\
=========

%SCRIPT plambda f/fuji.npy 'x,gs dup vnorm /'|plambda 'x,ds -1 *'|PLEGEND_REVERSE=1 palette 1% 0 nice - f/fuji_rcurv.png -l f/fuji_rcurv_pal.png &
%SCRIPT plambda f/fuji_b.npy 'x,gf dup vnorm /'|plambda 'x,db -1 *'|blur c 1|PLEGEND_REVERSE=1 palette -0.2 0.2 nice - f/fuji_curv.png -l f/fuji_curv_pal.png &

\only<1>{
\includegraphics[width=0.9\textwidth]{f/fuji_rcurv.png}%
\includegraphics[width=0.1\textwidth]{f/fuji_rcurv_pal.png}
\small
{\color{blue}valleys:$\mathrm{curv}(u)\!\ge\!0\quad$}
{\color{red}ridges:$\mathrm{curv}(u)\!\le\!0\quad$}
{flat:$\mathrm{curv}(u)=\textrm{NaN}$}
}%
\only<2>{
\includegraphics[width=0.9\textwidth]{f/fuji_curv.png}%
\includegraphics[width=0.1\textwidth]{f/fuji_curv_pal.png}
}
\only<3>{
\includegraphics[width=0.9\textwidth]{fs/fuji_far.jpg}\\
\scriptsize
\color{darkgreen}curvature + retinex kernel (rivers and passes become easier to see)
}

\end{frame}

%\begin{frame}
%Curvature sign = convexity, concavity of level lines\\
%
%================
%\end{frame}

\begin{frame}
LIGHT-BASED METHODS\\
===================

2. Light-based methods\\
2.1. Shadows\\
2.2. Point-source shading (exact, linear approx.)\\
2.3. Cloudy-sky shading (exact, linear approx.)\\
2.4. Specular shading (Phong)\\
2.5. Reflective shading (Oren-Nayar)\\
2.6. SAR direct model\\

\vfill

\includegraphics[width=\linewidth]{fs/brdf.png}\\
\small\color{darkgreen}
different lighting models (for a single light source)

\end{frame}

\begin{frame}
SHADOWS, SHADING, REFLECTIONS\\
=============================

\includegraphics[width=\linewidth]{fs/lighting.jpg}
\end{frame}

\begin{frame}
SHADOWS\\
=======

%SCRIPT plambda f/fuji.npy "10 /" -o f/fuji_p.npy
%SCRIPT shadowcast 0.58 0.81 0    -M f/fuji_p.npy f/fuji_shad1.png &
%SCRIPT shadowcast 0.58 0.81 0.5  -M f/fuji_p.npy f/fuji_shad2.png &
%SCRIPT shadowcast 0.58 0.81 1    -M f/fuji_p.npy f/fuji_shad3.png &
%SCRIPT shadowcast 0.58 0.81 1.5  -M f/fuji_p.npy f/fuji_shad4.png &
%SCRIPT shadowcast 0.58 0.81 2    -M f/fuji_p.npy f/fuji_shad5.png &
%SCRIPT shadowcast 0.18 0.26 2.93 -M f/fuji_p.npy f/fuji_shad6.png &
%SCRIPT shadowcast 0.18 0.26 4.93 -M f/fuji_p.npy f/fuji_shad7.png &
%SCRIPT shadowcast -0.26 0.18 2.00 -M f/fuji_p.npy f/fuji_shad8.png &
%SCRIPT shadowcast -0.36 0.18 0.70 -M f/fuji_p.npy f/fuji_shad9.png &
%SCRIPT shadowcast -0.46 0.18 0.60 -M f/fuji_p.npy f/fuji_shad10.png &

\only<1>{\includegraphics[width=0.9\linewidth]{f/fuji_shad1.png}\tiny1/10}%
\only<2>{\includegraphics[width=0.9\linewidth]{f/fuji_shad2.png}\tiny2/10}%
\only<3>{\includegraphics[width=0.9\linewidth]{f/fuji_shad3.png}\tiny3/10}%
\only<4>{\includegraphics[width=0.9\linewidth]{f/fuji_shad4.png}\tiny4/10}%
\only<5>{\includegraphics[width=0.9\linewidth]{f/fuji_shad5.png}\tiny5/10}%
\only<6>{\includegraphics[width=0.9\linewidth]{f/fuji_shad6.png}\tiny6/10}%
\only<7>{\includegraphics[width=0.9\linewidth]{f/fuji_shad7.png}\tiny7/10}%
\only<8>{\includegraphics[width=0.9\linewidth]{f/fuji_shad8.png}\tiny8/10}%
\only<9>{\includegraphics[width=0.9\linewidth]{f/fuji_shad9.png}\tiny9/10}%
\only<10>{\includegraphics[width=0.9\linewidth]{f/fuji_shad10.png}\tiny10/10}%

\end{frame}

\begin{frame}
LAMBERTIAN SHADING (LINEAR APPROXIMATION)\hfill$\color{blue}\mathrm{sun}=(\alpha,\beta,\gamma)$\\
=========================================

%SCRIPT SHADOWX=0.58 SHADOWY=0.81  plambda f/fuji_p.npy "x,Sf"|qauto -p 1 - f/fuji_lam1.png &
%SCRIPT SHADOWX=0.18 SHADOWY=0.26  plambda f/fuji_p.npy "x,Sf"|qauto -p 1 - f/fuji_lam2.png &
%SCRIPT SHADOWX=-0.26 SHADOWY=0.18 plambda f/fuji_p.npy "x,Sf"|qauto -p 1 - f/fuji_lam3.png &
%SCRIPT SHADOWX=-0.36 SHADOWY=0.18 plambda f/fuji_p.npy "x,Sf"|qauto -p 1 - f/fuji_lam4.png &
%SCRIPT SHADOWX=-0.46 SHADOWY=0.18 plambda f/fuji_p.npy "x,Sf"|qauto -p 1 - f/fuji_lam5.png &

Gouraud: $\color{blue}I = \frac{\alpha u_x + \beta u_y + \gamma}{\sqrt{1+u_x^2+u_y^2}}$\hfill
Linearized: $\color{blue}I = \alpha u_x + \beta u_y + \gamma$
\only<1>{\includegraphics[width=0.9\linewidth]{f/fuji_lam1.png}\tiny1/5}%
\only<2>{\includegraphics[width=0.9\linewidth]{f/fuji_lam2.png}\tiny2/5}%
\only<3>{\includegraphics[width=0.9\linewidth]{f/fuji_lam3.png}\tiny3/5}%
\only<4>{\includegraphics[width=0.9\linewidth]{f/fuji_lam4.png}\tiny4/5}%
\only<5>{\includegraphics[width=0.9\linewidth]{f/fuji_lam5.png}\tiny5/5}%

\end{frame}

%\begin{frame}
%LAMBERTIAN SHADING (LINEAR APPROXIMATION)\hfill$\color{blue}\mathrm{sun}=(\alpha,\beta,\gamma)$\\
%=========================================
%\begin{frame}

\begin{frame}
SPECULAR SHADING (PHONG)\\
========================\\
$\color{blue}g_\sigma(r\cdot v)\qquad\qquad
\color{blue}r = 2(s\cdot n)n-s$\\
{\tiny v=point of view, s=sun direction, n=surface normal}

\includegraphics[width=0.5\linewidth]{fs/phong.jpeg}\\
\includegraphics[width=0.9\linewidth]{fs/phongonly.png}

\end{frame}

\begin{frame}
Clear sky shading (ao sampling, + linear approximation)\\
================

%SCRIPT periodize < f/pteri.tif|fft|plambda ':R .7 ^ *'|fft -1|imhalve|qeasy -140 40 - f/pteri_linssao.png

%SCRIPT CX=1; for i in `seq 1 1000`; do SRAND=$i plambda -c "randu randu randu vec 2 * 1 - >1 <1 vnorm 1 > <1 dup vnorm / nan if" ; done|grep -v nan|grep -v '^-'|awk '{print $2 " "  $3 " " $1 }'|while read p q r; do echo shadowcast -M $p $q $r f/pteri.tif /tmp/teri_shey_${CX}.png ; CX=$[CX+1] ; done|parallel -j 8
%SCRIPT veco avg /tmp/teri_shey_* | qauto -p 1 - f/pteri_ao.png
\only<1>{\includegraphics[width=\textwidth]{f/pteri_linssao.png}}%
\only<2>{\includegraphics[width=\textwidth]{f/pteri_ao.png}}%
\end{frame}


\begin{frame}
SAR direct model\\
================

%SCRIPT plambda f/fuji.npy '30 /'|sarsim -89|qauto -p 1 - f/fuji_sar89.png &
%SCRIPT plambda f/fuji.npy '30 /'|sarsim -60|qauto -p 1 - f/fuji_sar60.png &
%SCRIPT plambda f/fuji.npy '30 /'|sarsim -45|qauto -p 1 - f/fuji_sar45.png &
%SCRIPT plambda f/fuji.npy '30 /'|sarsim -30|qauto -p 1 - f/fuji_sar30.png &
%SCRIPT plambda f/fuji.npy '10 /'|sarsim -30|qauto -p 1 - f/fuji_sar10.png &
%SCRIPT plambda f/fuji.npy '5 /'|sarsim -30|qauto -p 1 - f/fuji_sar5.png &

\only<1>{\includegraphics[width=\textwidth]{f/fuji_sar89.png}\tiny89}%
\only<2>{\includegraphics[width=\textwidth]{f/fuji_sar60.png}\tiny60}%
\only<3>{\includegraphics[width=\textwidth]{f/fuji_sar45.png}\tiny45}%
\only<4>{\includegraphics[width=\textwidth]{f/fuji_sar30.png}\tiny30}%
\only<5>{\includegraphics[width=\textwidth]{f/fuji_sar10.png}\tiny10}%
\only<1>{\includegraphics[width=\textwidth]{f/fuji_sar5.png}\tiny5}%
\end{frame}

\begin{frame}
FANCIER THINGS\\
==============\\
\small
combine several models (palette, diffuse, shadows, clearsyky)
\only<1>{\includegraphics[width=0.8\textwidth]{fs/reunion_combi1}}%

\end{frame}

%\begin{frame}
%To observe defects, understand geometry, etc\\
%================
%\end{frame}
%
%\begin{frame}
%Combined methods: tinted hillshade with level-lines\\
%================
%\end{frame}
%
%\begin{frame}
%As a direct model for several famous inverse problems\\
%================
%\end{frame}

\begin{frame}
TAKE HOME MESSAGE\\
=================

Bring your own hillshader

Light source is always up, never down
\end{frame}


\begin{frame}
COLOPHON\\
========

this document is a notebook
\end{frame}


%SCRIPT wait

\end{document}


% vim:sw=2 ts=2 spell spelllang=en:
