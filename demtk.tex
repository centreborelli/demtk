\title{A Toolkit for Digital Elevation Models}

\section{Sources of DEM}

%RUN_VERBATIMS python3

A DEM is a two-dimensional array of numbers.  Each number
represents a height in meters.  The positions inside the array are mapped into
geographical coordinates (typically UTM).  The mapping between positions in
the array and geographical coordinates is called \emph{georeferencing}.
Thus, a geo-referenced DEM can be interpreted as a cloud of points in 3D
space.

You can display the data in a DEM as an image, by mapping the heights to
pixel intensities:

\begin{quote}
\begin{verbatim}
import iio
x = iio.read("i/terrassa.tif")
X = (255.0 * (x - x.min())/(x.max() - x.min())).clip(0,255)
iio.write("o/terrassa.png", X)
\end{verbatim}
\includegraphics{o/terrassa.png}~\verb+terrassa.png+
\end{quote}

Notice that DEMs are stored as TIFF files of floating-point values.  To
display them, we quantize their range of values into 8 bits and save them on
a PNG file.




srtm4

lidar sites


\verb+dem_from_pair+

lonlat vs utm in DEM


\clearpage
\section{DEM Rendering}

%grayscale

The simplest way to represent a DEM is by mapping the heights to pixel
intensities of a traditional 8-bit grayscale image.
For that we will use the function \verb+qauto+, defined below
\begin{quote}
\begin{verbatim}
def qauto(x):   # quantize a floating-point image into 8 bits
        from numpy import uint8
        m = x.min()
        M = x.max()
        X  = (255 * (x.astype(float) - m) / (M - m) ).astype(uint8)
        return X
\end{verbatim}
\end{quote}
In the following experiments we work with the DEM stored on file
\verb+fuji.tif+ of the area around the Fuji volcano.  The data comes from
the SRTM4, which has a resolution of 90m per pixel.

\begin{quote}
\begin{verbatim}
import iio
x = iio.read("i/fuji.tif")
X = qauto(x)
iio.write("o/fuji.png", X)
\end{verbatim}
\includegraphics{o/fuji.png}~\verb+fuji.png+
\end{quote}

This image shows Fuji in the center (about $3700m$), surrounded by a mostly
flat area and a few mountains ranges around it (of maximum height
about~$1700m$).  When my son looked at this image he said it was a beautiful
picture of the moon, behind some clouds on a foggy night.  Clearly, images
obtained by mapping heights to intensities are difficult to interpret.


\subsection{Hillshading}

Our visual system is not used to the direct representation of geometry by
light intensity.
Instead, light is reflected by the surface of objects and we
receive this reflected light, which we know how to interpret.  The amount of
reflected light depends on the characteristics of the light and of the
surface.  The simplest
illumination model is the Lambertian lighting, whereby there is a single
point light source and each surface element reflects an amount of light
proportional to the scalar product between the surface normal and the
direction of the light source.

Consider the Lambertian model on a continuous setting, where the surface is
defined by a function~\[z=u(x,y)\] and the light source is located at the
infinity on the direction of the vector~$\vec s = (a,b,c)$.  The normal to
the surface at the location~$(x,y)$ is
\[
	\vec n(x,y) = \frac{1}{\sqrt{1+u_x^2+u_y^2}}\left(-u_x, -u_y, 1\right)
\]
and thus the light intensity reflected at the location~$(x,y)$ is
\[
	I(x,y) = \max\left(0, 
		\frac{c - au_x - bu_y}{\sqrt{1+u_x^2+u_y^2}}
	\right)
\]
The maximum is needed because when the scalar product~$\vec n\cdot\vec s$ is
negative, the surface element is invisible from the light source.  This
function~$I(x,y)$ is the light intensity that we would observe if we were
right in the vertical direction and very far away (this is called a
\emph{nadir view} or, in some contexts, an~\emph{orthoimage}).  Now, if~$c$
is large enough---the sun is way above the horizon---then there are no
shadows and~$I>0$.  Even more, if the slope of the surface is small with
respect to~$c$, then we can make the
approximation~$1\approx\sqrt{1+u_x^2+u_y^2}$.  In that case, the value of~$I$
is proportional to the directional derivative of~$u$ along the
direction~$(a,b)$.  Let us see how this looks.

First, we define a function to compute the directional derivative (by
default, from the top-left direction).

\begin{quote}
\begin{verbatim}
def render_shading(a, s=(1,1)):  # directional derivative of a along s
        from numpy import pad
        x = pad(a, ((0,0),(0,1)), 'edge')[:,1:] - a     # da / dx
        y = pad(a, ((0,1),(0,0)), 'edge')[1:,:] - a     # da / dy
        z = s[0] * x + s[1] * y                         # da / ds
        return z
\end{verbatim}
\end{quote}


And now we display the Fuji DEM using this function


%import iio
\begin{quote}
\begin{verbatim}
x = iio.read("i/fuji.tif").squeeze()
X = render_shading(x)
iio.write("o/fuji_shading.png", qauto(X) )
\end{verbatim}
\includegraphics{o/fuji_shading.png}~\verb+fuji_shading.png+
\end{quote}

Now, when my son saw this image, he said: well, now that's a close-up photo of a
crater in the moon!  He did not realize it was exactly the same data, but
with a different color scheme.

This image looks a bit flat to my eyes.
As if a slightly crumpled, but mostly flat, aluminium foil.
It can be made livelier by filtering it by a Riesz kernel, defined by
in the frequency domain by
\[
	\widehat{R_\sigma}(\xi,\eta)
	=
	\left(\xi^2+\eta^2\right)^{-\sigma/2}
\]

\begin{quote}
\begin{verbatim}
def filter_riesz0(x, s):   # Riesz scale-space (periodic boundary conditions)
        from numpy.fft import fft2, ifft2, fftfreq
        from numpy import repeat, hypot
        h, w = x.shape
        p = repeat(w*fftfreq(w).reshape(1,w), h, axis=0)  # x-frequencies
        q = repeat(h*fftfreq(h).reshape(h,1), w, axis=1)  # y-frequencies
        r = hypot(p, q)       # image of spectral radius
        r[0,0] = 1            # avoid warnings when 1/0
        X = fft2(x) * r**s    # apply the filter in the frequency domain
        if (s <= 0):
                X[0,0] = 0        # for negative s, set the mean to zero
        return ifft2(X).real
\end{verbatim}
\end{quote}

\begin{quote}
\begin{verbatim}
def filter_riesz(x, s):   # Riesz scale-space (symmetric boundary conditions)
        from numpy import pad
        h, w = x.shape
        y = pad(x, ((0,h),(0,w)), 'symmetric')
        z = filter_riesz0(y, s)
        return z[0:h,0:w]
\end{verbatim}
\end{quote}

\begin{quote}
\begin{verbatim}
x = iio.read("i/fuji.tif").squeeze()
X = filter_riesz(render_shading(x), -0.5)
iio.write("o/fuji_smooth.png", qauto(X) )
\end{verbatim}
\includegraphics{o/fuji_smooth.png}~\verb+fuji_smooth.png+
\end{quote}

Looking at this image, my kid said: this is the same crater as before, but
much closer!



\clearpage
\subsection{Shadows}

The technique of~\emph{hillshading} can be interpreted in two ways.
From the point of view of mathematics, it is simply a directional derivative
of the DEM re-scaled to a gray-scale palette.
From the point of view of computer graphics, it is a rendering of the DEM
data as a Lambertian (matte) surface, where the intensity of reflected light
by a surface element is proportional to the cosine of the angle towards the
light source.

The Lambertian model is one of the simpler lighting models, but not the
simplest.  The simplest one is a binary mask, depending on whether each
surface element is accessible by the light source.  Computing this
illumination model is called~\emph{shadow casting}.  Since the shadow casting
algorithm is not very interesting in itself, we refer to the~\verb+demtk+
library.

\begin{quote}
\begin{verbatim}
# cast shadows over Fuji
x = iio.read("i/fuji.tif").squeeze()
import demtk
z = demtk.render_shadows(x, (1,1,25))
iio.write("o/fuji_shadows.png", qauto(z) )
\end{verbatim}
\includegraphics{o/fuji_shadows.png}~\verb+fuji_shadows.png+
\end{quote}

While shadows alone are not very interesting by themselves, they can be used
to add some ``spice'' to other renderings.  More importantly, they form the
basis of the ambient occlusion introduced below.


\clearpage
\subsection{Ambient occlusion}

Deep valleys, narrow streets and the bottom of wells have a dark feeling
associated to them.  Even if they can be lit by direct sunlight, this is far
less likely than, for example, the top of a mountain.  A way to capture this
darkness is called ``ambient occlusion'' or the ``cloudy sky model''.  The
idea is that the brightness reflected by a surface element is proportional to
the spherical angle of visible sky from that point.  Thus the top of a
mountain sees the whole sky and it is the brightest possible, and the bottom
of a well sees just a tiny portion of the sky and it is very dark.  An exact
computation of ambient occlusion is very expensive, however it can be
very well approximated by sampling the sky and a finite set of points and
computing the average of the resulting shadows.


Notice that the average image of a small number of shadow images already
gives an idea of the overall geometry: you can see the bright ridges, the
dark valleys.
\begin{quote}
\begin{verbatim}
# coarse approximation of ambient occlusion
x = iio.read("i/fuji.tif").squeeze()
z = 0*x
z += demtk.render_shadows(x, ( 1, 0, 25) )
z += demtk.render_shadows(x, ( 1, 1, 25) )
z += demtk.render_shadows(x, ( 0, 1, 25) )
z += demtk.render_shadows(x, (-1, 1, 25) )
z += demtk.render_shadows(x, (-1, 0, 25) )
z += demtk.render_shadows(x, (-1,-1, 25) )
z += demtk.render_shadows(x, ( 0,-1, 25) )
z += demtk.render_shadows(x, ( 1,-1, 25) )
iio.write("o/fuji_aaaprox.png", qauto(z/8) )
\end{verbatim}
\includegraphics{o/fuji_aaaprox.png}~\verb+fuji_aaaprox.png+
\end{quote}

For urban regions, the average shadow image gives a clarifying view of dark
thin streets and bright rooftops.
\begin{quote}
\begin{verbatim}
# coarse approximation of ambient occlusion
x = iio.read("i/terrassa.tif").squeeze()
z = 0*x
z += demtk.render_shadows(x, ( 1, 0, 2) )
z += demtk.render_shadows(x, ( 1, 1, 2) )
z += demtk.render_shadows(x, ( 0, 1, 2) )
z += demtk.render_shadows(x, (-1, 1, 2) )
z += demtk.render_shadows(x, (-1, 0, 2) )
z += demtk.render_shadows(x, (-1,-1, 2) )
z += demtk.render_shadows(x, ( 0,-1, 2) )
z += demtk.render_shadows(x, ( 1,-1, 2) )
iio.write("o/terrassa_aaaprox.png", qauto(z/8) )
\end{verbatim}
\includegraphics{o/terrassa_aaaprox.png}~\verb+terrassa_aaaprox.png+
\end{quote}

Let us try a linear approximation of ambient occlusion.  If there is an object
of height~$h$ above me at a distance~$d$, it occludes a portion of the sky of
size proportional to $h/d$.  Of course, objects occlude themselves so this
effect is not additive.  But if we consider that it is additive---as a first
approximation---, this amounts to computing the convolution of the height
map~$h(x,y)$ by the kernel~$1/\sqrt{x^2+y^2}$, which is a particular case
for~$\sigma=1$ of our old friend: Riesz kernel.  Since we are only interested
in the substractive effect of heights (a deep well nearby does not make my
sky any brighter), we have to keep only the negative part of the Riesz
filter:

\begin{quote}
\begin{verbatim}
# linear approximation of ambient occlusion
x = iio.read("i/terrassa.tif").squeeze()
z = filter_riesz(x, 1)
iio.write("o/terrassa_aalin.png", qauto(z.clip(-1000,0)) )
\end{verbatim}
\includegraphics{o/terrassa_aalin.png}~\verb+terrassa_aalin.png+
\end{quote}



\clearpage
\subsection{Color palette}

A common way to display topographic maps is using a color palette.
To each height it corresponds a different color.  This is computed by
composing the height function with the image of the desired colormap.  The
colormap itself may be designed by hand, or read from ``palette'' image.
The traditional colormap goes from green to white through a sequence of
earthy tones.

\begin{quote}
\begin{verbatim}
# read palette from file
img_terrain = iio.read("i/DEM_poster.png")
pal_terrain = img_terrain[0][0:256]
\end{verbatim}
\includegraphics{i/DEM_poster.png}~\verb+DEM_poster.png+
\end{quote}

\begin{quote}
\begin{verbatim}
# apply palette to DEM
x = iio.read("i/fuji.tif").squeeze()
X = pal_terrain[qauto(x)]
iio.write("o/fuji_palette.png", X )
\end{verbatim}
\includegraphics{o/fuji_palette.png}~\verb+fuji_palette.png+
\end{quote}

Palette-rendered topographic maps are beautiful and colorful, but they are so
evocative that there is often the danger of interpreting them too seriously.
For instance, you see that there is a height where the green color
disappears, and another height where everything is white.  But this has
nothing to do with the tree-line or the presence of snow.  The assignment of
colors is essentially arbitrary.


\begin{samepage}
The arbitrariety of color palettes is very apparent when we use a
geographic palette to render a urban elevation map.  This does not mean that
the tops of the skyscrapers are covered in snow!

\begin{quote}
\begin{verbatim}
# render a urban DEM using a "terrain" palette
x = iio.read("i/terrassa.tif").squeeze()
X = pal_terrain[qauto(x)]
iio.write("o/terrassa_pal.png", X )
\end{verbatim}
\includegraphics{o/terrassa_pal.png}~\verb+terrassa_pal.png+
\end{quote}
\end{samepage}


Notice that, even if palettes provide an absolute meaning to heights, some
resolution is lost.  For example, it is almost impossible to see the concave
crater of the volcano, while it was very clearly seen on the hill-shaded DEM.
Thus, a useful technique consists in combining both renderings, for example
by taking their average:

\begin{quote}
\begin{verbatim}
x = iio.read("i/fuji.tif").squeeze()
from numpy import newaxis
x_lam = qauto(render_shading(x))[:,:,newaxis]  # lambertian shading of x
x_pal = pal_terrain[qauto(x)]     # color palette of x
X = qauto(x_lam**.8 * x_pal)          # combination (geometric mean)
iio.write("o/fuji_combined.png", X)
\end{verbatim}
\includegraphics{o/fuji_combined.png}~\verb+fuji_combined.png+
\end{quote}



%\subsection{Level lines}
%
%

\clearpage
\subsection{Curvature map}

%sign of Riesz scale space as an indicator of ridges/valleys
The sign of Riesz scale space is a good indicator of local
convexity/concavity.  By using oriented palettes we can highlight the valleys
(blue)
and ridges (red) of a topographic map.  

\begin{quote}
\begin{verbatim}
from numpy import fmax, dstack
x = iio.read("i/fuji.tif").squeeze()
r = filter_riesz(x, 1)
x_signed  = qauto(r)
x_ridges  = 2.0*qauto(fmax(0,r)).clip(0,255)
x_valleys = 255-2.0*qauto(fmax(0,-r)).clip(0,255)
x_curv    = dstack([x_valleys, x_valleys-x_ridges, 255.0-x_ridges])
iio.write("o/fuji_signed.png" , x_signed )
iio.write("o/fuji_ridges.png" , x_ridges )
iio.write("o/fuji_valleys.png", x_valleys )
iio.write("o/fuji_curv.png", x_curv )
\end{verbatim}
\includegraphics{o/fuji_signed.png}~\verb+fuji_signed.png+
\includegraphics{o/fuji_ridges.png}~\verb+fuji_ridges.png+
\includegraphics{o/fuji_valleys.png}~\verb+fuji_valleys.png+
\includegraphics{o/fuji_curv.png}~\verb+fuji_curv.png+
\end{quote}

For urban regions, this gives a segmentation between rooftops (in red),
streets (in blue) and flat or otherwise ambiguous parts (in white).

%\begin{quote}
%\begin{verbatim}
%# build blue-yellow palette
%from numpy import arange, vstack
%pal_blue_yellow = vstack([arange(256), arange(256), 255-arange(256)]).T
%
%# render urban scene with sign of a Riesz filter
%from numpy import uint8
%x = iio.read("i/terrassa.tif").squeeze()
%t = filter_riesz(x, 1)
%T = (127.5 + 60 * t / t.std()).clip(0.255).astype(uint8)
%iio.write("o/terrassa_curv.png", pal_blue_yellow[T] )
%\end{verbatim}
%\includegraphics{o/terrassa_curv.png}~\verb+terrassa_curv.png+
%\end{quote}

\clearpage
\subsection{Combined rendering}

The best results are often obtained by merging the outputs of several
techniques above.  The function \verb+demtk.render+ by default produces a
high-contrast rendering suitable for display in low-resolution projectors.

\begin{quote}
\begin{verbatim}
import demtk, iio
x = iio.read("i/terrassa.tif").squeeze()
iio.write("o/terrassa_full.png", demtk.render(x))

x = iio.read("i/fuji.tif").squeeze()
iio.write("o/fuji_full.png", demtk.render(x))
\end{verbatim}
\includegraphics{o/terrassa_full.png}~\verb+terrassa_full.png+
\includegraphics{o/fuji_full.png}~\verb+fuji_full.png+
\end{quote}



\section{DEM Denoising}

\begin{quote}
\begin{verbatim}
import demtk, iio
x = iio.read("i/terrassa.tif").squeeze()
iio.write("o/terrassa_erosion.png",  demtk.render(demtk.cross_erosion(x)))
iio.write("o/terrassa_dilation.png", demtk.render(demtk.cross_dilation(x)))
iio.write("o/terrassa_median.png", demtk.render(demtk.cross_median(x)))
iio.write("o/terrassa_median5.png", demtk.render(demtk.cross_median(x,5)))
\end{verbatim}
\includegraphics{o/terrassa_erosion.png}~\verb+terrassa_erosion.png+
\includegraphics{o/terrassa_dilation.png}~\verb+terrassa_dilation.png+
\includegraphics{o/terrassa_median5.png}~\verb+terrassa_median5.png+
\end{quote}

\begin{quote}
\begin{verbatim}
import demtk, iio
x = iio.read("i/fuji.tif").squeeze()
iio.write("o/fuji_erosion.png",  demtk.render(demtk.cross_erosion(x)))
iio.write("o/fuji_dilation.png", demtk.render(demtk.cross_dilation(x)))
\end{verbatim}
\includegraphics{o/fuji_erosion.png}~\verb+fuji_erosion.png+
\includegraphics{o/fuji_dilation.png}~\verb+fuji_dilation.png+
\end{quote}

morphological operations

cc filter


\section{DEM Interpolation}

cc border, avg, min, min5pc

poisson, biharmonic, TV, with Dirichlet boundary conditions

neumann boundaries at selected points


\section{DEM Registration}

apply shift (manually selected)

multiscale correlation

phase correlation


\section{DEM Comparison}


difference of registered images, signed palette

filtering of difference


\section{DEM Fusion}

nanavg

nanmed

xmedians

cnt

filtered fusion (by iqd, cnt, ...)


\section{DEM Flattening}

plyflatten

multiscale plyflatten

splatting


\section{DEM Elevation}

isolated points

local mesh

refined mesh heuristics


\section{DEM Colorization}

naive colorization

shadowed colorization

orthophoto merging (color median)

orthophoto merging by Poisson

orthophoto merging by osmosis



% vim:set tw=77 filetype=tex spell spelllang=en:
